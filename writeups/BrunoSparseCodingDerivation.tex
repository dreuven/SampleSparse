\documentclass{article}
\usepackage[utf8]{inputenc}
\usepackage{listings}
\title{BrunoSparseCodingDerivation}
\author{dreuven13 }
\date{June 2016}

\begin{document}

\maketitle

\section{Introduction}
Here is the code to get us started. Courtesy of Mudigonda et. al:
\begin{lstlisting}
    tmp = (self.data - self.basis.dot(coeff))**2
    tmp = 0.5*tmp.sum(axis=0)
    tmp = tmp.mean()
    sparsity = self.lam * T.abs(coeff).sum(axis=0).mean()
    obj = tmp + sparsity
\end{lstlisting}\\
\newcommand{\ErrorFunc}{\sum_{j=1}^{j=p}(I_j_i - \sum_{l=1}^{l=k}\Phi_j_l * a_l_i)}
\newcommand{\ErrFSinJ}{(I_j_i - \sum_{l=1}^{l=k}\Phi_j_l * a_l_i)}
The Energy Function from the code in sum notation is:
\begin{equation}
    E = -1 * \frac{1}{2N} *  \sum_{i=1}^{i=N}\ErrorFunc^2  + \lambda * \frac{1}{N} *  \sum_{i=1}^{i=N}\sum_{j=1}^{j=k}|a_j_i|
\end{equation}
Energy Gradient W/Respect to Phi: \\
\begin{equation}
    \frac{dE}{d\Phi_x_y} = \frac{1}{N}\sum_{i=1}^{i=N}\ErrFSinJ * a_y_i
\end{equation}
Note:  \[\frac{dE}{d\Phi_x_y}\] only non-zero when j = x and l = y. Thus, we can remove the sum over values of j and consider only j = x.
In Plain English:\\
\[\frac{dE}{d\Phi_x_y}\] is the update for the x\textsuperscript{th} pixel of the y\textsuperscript{th} receptive field, which can be found by summing over the reconstruction error across all input images at the x\textsuperscript{th} pixel and weighting this error by the strength of the y\textsuperscript{th} pixel phi coefficient for that sample.\\
In Matrix Form this is:
\begin{equation}
    -1 * \frac{1}{N}(I - \Phi * a)a^T
\end{equation}
\\
Energy Gradient W/Respect to a:
\begin{equation}
    \frac{dE}{da_x_y} = -1 * \frac{1}{N}\sum_{i=1}^{i=N}\ErrorFunc*\Phi_j_i + \frac{1}{N}\lambda * sign(a_x_y)
\end{equation}
Note: The only place where the sum is non-zero is where l = x and i = y. Thus we can ignore the $\frac{1}{N}$ since it is just $\frac{1}{1}$ and the summation over i, which is just the sum of one value of i. This becomes:
\begin{equation}
    \frac{dE}{da_x_y} = -1 * [\sum_{j=1}^{j=p}I_j_y - \sum_{l=1}^{l=k}\Phi_j_l* a_l_y]*\Phi_j_x + \lambda * sign(a_x_y)
\end{equation}
This equation makes intuitive sense when considering what $\frac{dE}{da_x_y}$ means. It is finding the derivative for the coefficient of the x\textsuperscript{th} basis function in the y\textsuperscript{th} sample. This is why we don't sum over all of the samples and this is why we weight the reconstruction error by the activation of the x\textsuperscript{th} phi. In matrix notation this amounts to:
\begin{equation}
    \frac{dE}{dA} = -1 * ((I - \Phi*a)^T*\Phi)^T + \lambda*sign(a)
\end{equation}
\end{document}

